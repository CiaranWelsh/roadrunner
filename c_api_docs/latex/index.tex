\hypertarget{index_intro_sec}{}\section{Introduction}\label{index_intro_sec}
Road\+Runner is a S\+B\+M\+L compliant high performance and portable simulation engine for systems and synthetic biology. To run a simple S\+B\+M\+L model and generate time series data we would write the following code\+:


\begin{DoxyCode}
\textcolor{preprocessor}{#undef \_\_cplusplus}
\textcolor{preprocessor}{#define STATIC\_RRC}
\textcolor{preprocessor}{#include <stdio.h>}
\textcolor{preprocessor}{#include <stdlib.h>}
\textcolor{preprocessor}{#include "\hyperlink{rrc__api_8h}{rrc\_api.h}"}
\textcolor{preprocessor}{#include "\hyperlink{rrc__types_8h}{rrc\_types.h}"}
\textcolor{preprocessor}{#include "\hyperlink{rrc__utilities_8h}{rrc\_utilities.h}"}
\textcolor{keywordtype}{int} main (\textcolor{keywordtype}{int} argc, \textcolor{keywordtype}{char} *argv[]) \{
    \hyperlink{rrc__types_8h_a1d68f0592372208fa5a5f2799ea4b3ae}{RRHandle} rrHandle;
    \hyperlink{struct_r_r_c_data}{RRCDataPtr} result;

    printf (\textcolor{stringliteral}{"Starting Test Program %s\(\backslash\)n"}, argv[0]);
    rrHandle = \hyperlink{group__initialization_gac5b4dbf9108edb1ea6e315379e5576e0}{createRRInstance}();
    \textcolor{keywordflow}{if} (!\hyperlink{group__loadsave_ga275b8f8d7350505c383fdc9634713041}{loadSBMLFromFile} (rrHandle, \textcolor{stringliteral}{"feedback.xml"})) \{
        printf (\textcolor{stringliteral}{"Failed to load model: %s\(\backslash\)n"}, \hyperlink{group__errorfunctions_ga97659609c7c7715bd37e0962efd21c3a}{getLastError} ());
        getchar ();
        exit (0);
    \}
    result = \hyperlink{group__simulation_gaa568722adbce33e145ce8c4a78146465}{simulateEx} (rrHandle, 0, 10, 100);
    printf (\hyperlink{group__to_string_ga505aeb10e21f0ef93770f7723aa12514}{rrCDataToString} (result));
    \hyperlink{group__free_routines_gaf088e47d2a725b40685364cd99a574f2}{freeRRCData}(result);

    getchar ();
    exit (0);
\}
\end{DoxyCode}


More complex example, using C A\+P\+I\+: 
\begin{DoxyCode}
\textcolor{preprocessor}{#undef \_\_cplusplus}
\textcolor{preprocessor}{#define STATIC\_RRC}
\textcolor{preprocessor}{#include <stdio.h>}
\textcolor{preprocessor}{#include <stdlib.h>}
\textcolor{preprocessor}{#include "\hyperlink{rrc__api_8h}{rrc\_api.h}"}
\textcolor{preprocessor}{#include "\hyperlink{rrc__types_8h}{rrc\_types.h}"}
\textcolor{preprocessor}{#include "\hyperlink{rrc__utilities_8h}{rrc\_utilities.h}"}
\textcolor{keywordtype}{int} main (\textcolor{keywordtype}{int} argc, \textcolor{keywordtype}{char} *argv[]) \{
   \hyperlink{rrc__types_8h_a1d68f0592372208fa5a5f2799ea4b3ae}{RRHandle} rrHandle;
   \hyperlink{struct_r_r_c_data}{RRCDataPtr} result;
   \textcolor{keywordtype}{int} index;
   \textcolor{keywordtype}{int} col;
   \textcolor{keywordtype}{int} row;
   printf (\textcolor{stringliteral}{"Starting Test Program %s\(\backslash\)n"}, argv[0]);
   rrHandle = \hyperlink{group__initialization_gac5b4dbf9108edb1ea6e315379e5576e0}{createRRInstance}();
   \textcolor{keywordflow}{if} (!\hyperlink{group__loadsave_gafb3515fd8472da389f5f24530017f037}{loadSBML} (rrHandle, \textcolor{stringliteral}{"feedback.xml"})) \{
      printf (\textcolor{stringliteral}{"Error while loading SBML file\(\backslash\)n"});
      printf (\textcolor{stringliteral}{"Error message: %s\(\backslash\)n"}, \hyperlink{group__errorfunctions_ga97659609c7c7715bd37e0962efd21c3a}{getLastError}());
      getchar ();
      exit (0);
   \}
   result = \hyperlink{group__simulation_gaa568722adbce33e145ce8c4a78146465}{simulateEx} (rrHandle, 0, 10, 10);  \textcolor{comment}{// start time, end time, and number of points}
   index = 0;
   \textcolor{comment}{// Print out column headers... typically time and species.}
   \textcolor{keywordflow}{for} (col = 0; col < result->\hyperlink{struct_r_r_c_data_a17c9a5894aa9cb3789346dcaa9c370bb}{CSize}; col++)
   \{
      printf (\textcolor{stringliteral}{"%10s"}, result->\hyperlink{struct_r_r_c_data_ab339159e5604808f92fe793f4f43da03}{ColumnHeaders}[index++]);
      \textcolor{keywordflow}{if} (col < result->CSize - 1)
      \{
         printf (\textcolor{stringliteral}{"\(\backslash\)t"});
      \}
   \}
   printf (\textcolor{stringliteral}{"\(\backslash\)n"});
   index = 0;
   \textcolor{comment}{// Print out the data}
   \textcolor{keywordflow}{for} (row = 0; row < result->\hyperlink{struct_r_r_c_data_a4d8512c879223c0e0d1522dae38e7819}{RSize}; row++)
   \{
      \textcolor{keywordflow}{for} (col = 0; col < result->\hyperlink{struct_r_r_c_data_a17c9a5894aa9cb3789346dcaa9c370bb}{CSize}; col++)
      \{
         printf (\textcolor{stringliteral}{"%10f"}, result->\hyperlink{struct_r_r_c_data_a7c5cbda3aa940f4b0d6e8a1679307dfc}{Data}[index++]);
         \textcolor{keywordflow}{if} (col < result->CSize -1)
         \{
            printf (\textcolor{stringliteral}{"\(\backslash\)t"});
         \}
      \}
   printf (\textcolor{stringliteral}{"\(\backslash\)n"});
   \}
   \textcolor{comment}{//Cleanup}
   \hyperlink{group__free_routines_gaf088e47d2a725b40685364cd99a574f2}{freeRRCData} (result);
   \hyperlink{group__initialization_gaa085b3245183be2ee16a94dfe0de1751}{freeRRInstance} (rrHandle);
   getchar ();
   exit (0);
\}
\end{DoxyCode}


Would create output as shown below\+:


\begin{DoxyCode}
Starting Test Program: <File path Here>
     time            [S1]            [S2]            [S3]            [S4]
 0.000000        0.000000        0.000000        0.000000        0.000000
 1.111111        3.295975        1.677255        1.121418        1.074708
 2.222222        0.971810        1.658970        1.841065        2.192728
 3.333333        0.137340        0.501854        1.295138        2.444883
 4.444445        0.141470        0.200937        0.549172        1.505662
 5.555556        1.831017        1.317792        1.129982        1.351300
 6.666667        0.306310        0.775477        1.304950        1.952076
 7.777778        0.193459        0.268986        0.628542        1.483161
 8.888889        1.566864        1.219950        1.105718        1.370199
10.000000        0.269437        0.678127        1.199353        1.868247
\end{DoxyCode}
 \hypertarget{index_install_sec}{}\section{Installation}\label{index_install_sec}
Installation documentation is provided at lib\+Road\+Runner.\+org.\hypertarget{index_license_sec}{}\section{License}\label{index_license_sec}
Copyright (C) 2012-\/2015 University of Washington, Seattle, W\+A, U\+S\+A

Licensed under the Apache License, Version 2.\+0 (the \char`\"{}\+License\char`\"{}); you may not use this file except in compliance with the License. You may obtain a copy of the License at \begin{DoxyVerb}http://www.apache.org/licenses/LICENSE-2.0
\end{DoxyVerb}


Unless required by applicable law or agreed to in writing, software distributed under the License is distributed on an \char`\"{}\+A\+S I\+S\char`\"{} B\+A\+S\+I\+S, W\+I\+T\+H\+O\+U\+T W\+A\+R\+R\+A\+N\+T\+I\+E\+S O\+R C\+O\+N\+D\+I\+T\+I\+O\+N\+S O\+F A\+N\+Y K\+I\+N\+D, either express or implied. See the License for the specific language governing permissions and limitations under the License.

In plain english this means\+:

You C\+A\+N freely download and use this software, in whole or in part, for personal, company internal, or commercial purposes;

You C\+A\+N use the software in packages or distributions that you create.

You S\+H\+O\+U\+L\+D include a copy of the license in any redistribution you may make;

You are N\+O\+T required include the source of software, or of any modifications you may have made to it, in any redistribution you may assemble that includes it.

Y\+O\+U C\+A\+N\+N\+O\+T\+:

redistribute any piece of this software without proper attribution; 